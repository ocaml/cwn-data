% Created 2024-03-05 Tue 15:35
% Intended LaTeX compiler: pdflatex
\documentclass[11pt]{article}
\usepackage[utf8]{inputenc}
\usepackage[T1]{fontenc}
\usepackage{graphicx}
\usepackage{longtable}
\usepackage{wrapfig}
\usepackage{rotating}
\usepackage[normalem]{ulem}
\usepackage{amsmath}
\usepackage{amssymb}
\usepackage{capt-of}
\usepackage{hyperref}
\usepackage{tikz-cd}
\date{\today}
\title{OCaml Weekly News}
\hypersetup{
 pdfauthor={Alan Schmitt},
 pdftitle={OCaml Weekly News},
 pdfkeywords={},
 pdfsubject={},
 pdfcreator={Emacs 29.1 (Org mode 9.7)}, 
 pdflang={English}}
\usepackage{biblatex}
\addbibresource{/Users/schmitta/projets/bib/alan.bib}
\begin{document}

\maketitle
\href{https://alan.petitepomme.net/cwn/2024.02.27.html}{Previous Week} \href{https://alan.petitepomme.net/cwn/index.html}{Up} \href{https://alan.petitepomme.net/cwn/2024.03.12.html}{Next Week}

Hello

Here is the latest OCaml Weekly News, for the week of February 27 to March 05, 2024.

\setcounter{tocdepth}{1}
\tableofcontents
\section*{mirage-crypto 0.11.3 (with more speed for elliptic curves) -- and the future roadmap of mirage-crypto}
\label{1}
Archive: \url{https://discuss.ocaml.org/t/ann-mirage-crypto-0-11-3-with-more-speed-for-elliptic-curves-and-the-future-roadmap-of-mirage-crypto/14200/1}
\subsection*{Hannes Mehnert announced}
\label{sec:org38175a6}


we're happy to announce mirage-crypto 0.11.3 (which just got merged to opam-repository), which includes huge performance improvements
for elliptic curves. The API didn't change at all :)

The background story is that we finally merged the \href{https://github.com/mirage/mirage-crypto/pull/146}{``use bytes instead of Cstruct.t''
PR} which was opened \texttt{2.5
years ago by @dinosaure.
We reviewed that, and did some
benchmarks. And even went a bit further and are now using
\textasciitilde{}string} (instead of \texttt{bytes}). See
\url{https://blog.robur.coop/articles/speeding-ec-string.html} for further details.

Another PR worth mentioning is \href{https://github.com/mirage/mirage-crypto/pull/191}{use windowed algorithm for base scalar
multiplication} from Virgile Robles -- now some precomputed tables are shipped
(same approach was done for 25519 already).

\href{https://github.com/mirage/mirage-crypto/pull/191}{See}
\href{https://github.com/mirage/mirage-crypto/pull/191\#issuecomment-1932003002}{this}
\href{https://github.com/mirage/mirage-crypto/pull/191\#issuecomment-1951836996}{PR} (and the [release notes](
\url{https://github.com/mirage/mirage-crypto/releases/tag/v0.11.3})) for some detailed performance numbers on different CPUs -- the P256
sign operation is around 10x faster than older releases. This is still 5 times slower than OpenSSL - but then we use
\href{https://github.com/mit-plv/fiat-crypto/}{fiat-crypto} instead of handcrafted assembly code. We're keen to improve the performance
even further -- ideas, observations, experiments and PRs are very welcome. We investigated benchmarking of e.g. \href{https://github.com/mirage/mirage-crypto/pull/202}{digest algorithms
across the OCaml ecosystem and OpenSSL as baseline} and welcome improvements and
further work on that (especially AES-GCM and Poly1305-ChaCha20 are painfully slow compared to OpenSSL).

Other improvements and fixes include support for Loongarch, NetBSD, use rdtime instead of rdcycle on RISC-V when in user mode,
initial support for CL.EXE. Thanks to everyone involved in this released: @jbeckford @reynir @dinosaure @palainp @edwin

The \href{https://github.com/mirage/mirage-crypto/releases/tag/v0.11.3}{full changelog} may be worth to read.
\subsubsection*{Future roadmap (breaking changes)}
\label{sec:org4a25ec4}

Also, please note if you're using mirage-crypto that we'll revise the API and no longer use Cstruct.t / bigarrays, but instead
bytes/string. 0.11.3 will be the last release using Cstruct.t. The hash functionality (\texttt{Mirage\_crypto.Hash}) will also be removed
(since \href{https://github.com/mirage/digestif}{digestif} implements them nicely). Please voice your concerns / ideas at
\url{https://github.com/mirage/mirage-crypto/issues/205}
\section*{Re: Ocsigen: summary of recent releases}
\label{2}
Archive: \url{https://discuss.ocaml.org/t/ocsigen-summary-of-recent-releases/13817/6}
\subsection*{Vincent Balat announced}
\label{sec:orgf73d38b}


Eliom 10.3.1 released:
\begin{itemize}
\item Fixing Problem with browser navigation \url{https://github.com/ocsigen/eliom/issues/781}
\item Adding raw events handlers in Eliom\_content.Html.F.Raw for server-side only programming
\end{itemize}
\section*{OCaml Platform Newsletter: January 2024}
\label{3}
Archive: \url{https://discuss.ocaml.org/t/ocaml-platform-newsletter-january-2024/14203/1}
\subsection*{Thibaut Mattio announced}
\label{sec:orgdbf7237}


Welcome to the ninth edition of the OCaml Platform newsletter!

In this January 2024 edition, we are excited to bring you the latest on the OCaml Platform, continuing our tradition of highlighting
recent developments as seen in \href{https://discuss.ocaml.org/tag/platform-newsletter}{previous editions}. To understand the direction
we're headed, especially regarding development workflows and user experience improvements, check out our
\href{https://ocaml.org/docs/platform-roadmap}{roadmap}.

\textbf{Highlights:}
\begin{itemize}
\item A preview version of the long-awaited Merlin project-wide references is available. Read more on \href{https://discuss.ocaml.org/t/ann-preview-play-with-project-wide-occurrences-for-ocaml/13814}{the announcement}.
\item A first beta of opam 2.2 is \href{https://ocaml.org/changelog/2024-01-18-opam-2-2-0-beta1}{available}! Try it, and let the opam team know if you encounter any issues using opam on Windows.
\item The \texttt{odoc} team started an effort to unify the OCaml.org package documentation with the local workflow provided by Dune. This is very exciting, as the result should be a much improved local documentation with Dune and faster releases of \texttt{odoc} features on OCaml.org. They are at the very beginning of the project, but stay tuned for exciting news in the coming months!
\end{itemize}

\textbf{Releases:}
\begin{itemize}
\item \href{https://ocaml.org/changelog/2024-01-16-dune-3.13.0}{Dune 3.13.0}
\item \href{https://ocaml.org/changelog/2024-01-05-dune-3.12.2}{Dune 3.12.2}
\item \href{https://ocaml.org/changelog/2024-01-18-opam-2-2-0-beta1}{opam 2.2.0\textasciitilde{}beta1}
\item \href{https://ocaml.org/changelog/2024-01-24-odoc-2.4.1}{\texttt{odoc} 2.4.1}
\end{itemize}
\subsubsection*{\textbf{{[}Dune]} Exploring Package Management in Dune (\href{https://ocaml.org/docs/platform-roadmap\#w4-build-a-project}{W4})}
\label{sec:org628352d}

\textbf{Contributed by:} @rgrinberg (Tarides), @Leonidas-from-XIV (Tarides), @gridbugs (Tarides), @kit-ty-kate (Tarides), @Alizter

\textbf{Why:} Unify OCaml tooling under a single command line for all development workflows. This addresses one of the most important pain
points \href{https://www.dropbox.com/s/omba1d8vhljnrcn/OCaml-user-survey-2020.pdf?dl=0}{reported by the community}.

\textbf{What:} Prototyping the integration of package management into Dune using opam as a library. We're introducing a \texttt{dune pkg lock}
command to generate a lock file and enhancing \texttt{dune build} to handle dependencies in the lock file. More details in the \href{https://github.com/ocaml/dune/issues/7680}{Dune
RFC}.

\textbf{Activities:}
\begin{itemize}
\item Support opam’s pin-depends field -- \url{https://github.com/ocaml/dune/pull/9685}
\item Set \%\{pkg:dev\} correctly for packages that use dev sources -- \url{https://github.com/ocaml/dune/pull/9605}
\item Remove Repository\_id refactor, which instead now uses Git URLs to specify revisions -- \url{https://github.com/ocaml/dune/pull/9614}
\item Remove \texttt{-{}-{}skip-update} and automatically infer offline mode when possible -- \url{https://github.com/ocaml/dune/pull/9683}
\item Support submodules in repos -- \url{https://github.com/ocaml/dune/pull/9798}
\item Don't download the same package source archive multiple times during a build. Many OCaml packages are in Git repos (and source archives) with several other related packages, and it's common for a project to depend on several packages from the same repo. Without this change, the source archive for a repo would be downloaded once for each package from that repo appearing in a project's dependencies -- \url{https://github.com/ocaml/dune/pull/9771}
\item Add a cond statement for choosing lockdirs. This allows the lockdir to be chosen based on properties of the current system (e.g., OS, architecture) which will simplify working on projects with system-specific dependencies. -- \url{https://github.com/ocaml/dune/pull/9750}
\end{itemize}
\subsubsection*{\textbf{{[}opam]} Native Support for Windows in opam 2.2 (\href{https://ocaml.org/docs/platform-roadmap\#w5-manage-dependencies}{W5})}
\label{sec:org59155c9}

\textbf{Contributed by:} @rjbou (OCamlPro), @kit-ty-kate (Tarides), @dra27 (Tarides), @AltGr (OCamlPro)

\textbf{Why:} Enhance OCaml's viability on Windows by integrating native opam and \texttt{opam-repository} support, fostering a larger community
and more Windows-friendly packages.

\textbf{What:} Releasing opam 2.2 with native Windows support, making the official \texttt{opam-repository} usable on Windows platforms.

\textbf{Activities:}
\begin{itemize}
\item Add \texttt{rsync} package to internal Cygwin packages list (enables local pinning and is used by the VCS backends -- \href{https://github.com/ocaml/opam/pull/5808}{ocaml/opam\#5808}
\item Check and advertise to use Git for Windows -- \href{https://github.com/ocaml/opam/pull/5718}{ocaml/opam\#5718}
\item Released \href{https://ocaml.org/changelog/2024-01-18-opam-2-2-0-beta1}{opam 2.2\textasciitilde{}beta1}
\end{itemize}
\subsubsection*{\textbf{{[}\textasciitilde{}odoc\textasciitilde{}]} Unify OCaml.org and Local Package Documentation}
\label{sec:orgf3402ff}
(\href{https://ocaml.org/docs/platform-roadmap\#w25-generate-documentation}{W25})

\textbf{Contributed by:} @jonludlam (Tarides), @julow (Tarides), @panglesd (Tarides)

\textbf{Why:} Improving local documentation generation workflow will help package authors write better documentation for their packages,
and consolidating the different \texttt{odoc} documentation generators will help make continuous improvements to \texttt{odoc} available to a
larger audience.

\textbf{What:} We will write conventions that drivers must follow to ensure that their output will be functional. Once established, we
will update the dune rules to follow these rules, access new \texttt{odoc} features (e.g., source rendering) and provide similar
functionalities to docs.ocaml.org (a navigational sidebar for instance). This will effectively make Dune usable to generate OCaml.org
package documentation.

\textbf{Activities:}
\begin{itemize}
\item We started by comparing the various drivers, their needs and constraints, and to flesh out what the conventions could look like. We will publish an RFC before starting the implementation work to ensure that we indeed understood the needs of everyone.
\end{itemize}
\subsubsection*{\textbf{{[}\textasciitilde{}odoc\textasciitilde{}]} Add Search Capabilities to \texttt{odoc} (\href{https://ocaml.org/docs/platform-roadmap\#w25-generate-documentation}{W25})}
\label{sec:org03e2b6f}

\textbf{Contributed by:} @panglesd (Tarides), @EmileTrotignon (Tarides), @julow (Tarides), @jonludlam (Tarides)

\textbf{Why:} Improve usability and navigability in OCaml packages documentation, both locally and on OCaml.org, by offering advanced
search options like type-based queries.

\textbf{What:} Implementing a search engine interface in \texttt{odoc}, complete with a UI and a search index. Additionally, we're developing a
default client-side search engine based on Sherlodoc.

\textbf{Activities:}
\begin{itemize}
\item We kept working on Sherlodoc in Januray, and a \href{https://discuss.ocaml.org/t/ann-sherlodoc-a-search-engine-for-ocaml-documentation/14011}{new version was released} a few weeks ago, which can now be embedded on \texttt{odoc}-built doc sites.
\item We also finished updating the Dune rules which drive \texttt{odoc}, to enable the new search feature on locally built docs. These changes were released as part of Dune 3.14.0. -- \href{https://github.com/ocaml/dune/pull/9772}{ocaml/dune\#9772}
\end{itemize}
\subsubsection*{\textbf{{[}\textasciitilde{}odoc\textasciitilde{}]} Syntax for Images and Assets in \texttt{odoc} (\href{https://ocaml.org/docs/platform-roadmap\#w25-generate-documentation}{W25})}
\label{sec:org982f96f}

\textbf{Contributed by:} @panglesd (Tarides), @jonludlam (Tarides), @dbuenzli, @gpetiot (Tarides)

\textbf{Why:} Empower package authors to create rich, engaging documentation by enabling the integration of multimedia elements directly
into OCaml package documentation.

\textbf{What:} We're introducing new syntax and support for embedding media (images, audio, videos) and handling assets within the \texttt{odoc}
environment.

\textbf{Activities:}
\begin{itemize}
\item The PR is still under active review and we're addressing the last minor concerns. -- \href{https://github.com/ocaml/odoc/pull/1002}{ocaml/odoc\#1002}
\end{itemize}
\subsubsection*{\textbf{{[}\textasciitilde{}odoc\textasciitilde{}]} Improving \texttt{odoc} Performance (\href{https://ocaml.org/docs/platform-roadmap\#w25-generate-documentation}{W25})}
\label{sec:org94b254b}

\textbf{Contributed by:} @jonludlam (Tarides), @julow (Tarides), @gpetiot (Tarides)

\textbf{Why:} Address performance issues in \texttt{odoc}, particularly for large-scale documentation, to enhance efficiency and user experience
and unlock local documentation generation in large code bases.

\textbf{What:} Profiling \texttt{odoc} to identify the main performance bottlenecks and optimising \texttt{odoc} with the findings.

\textbf{Activities:}
\begin{itemize}
\item We investigated a couple of issues brought forth by the \texttt{module type of} fix that was mentioned last month. This eventually resulted in a series of PRs: \href{https://github.com/ocaml/odoc/pull/1078}{ocaml/odoc\#1078}, \href{https://github.com/ocaml/odoc/pull/1079}{ocaml/odoc\#1079} and \href{https://github.com/ocaml/odoc/pull/1081}{ocaml/odoc\#1081}
\item We also noticed that \texttt{odoc}'s handling of the load path was quadratic, so we changed that in \href{https://github.com/ocaml/odoc/pull/1075}{ocaml/odoc\#1075}.
\end{itemize}
\subsubsection*{\textbf{{[}Merlin]} Support for Project-Wide References in Merlin (\href{https://ocaml.org/docs/platform-roadmap\#w19-navigate-code}{W19})}
\label{sec:org3b37b55}

\textbf{Contributed by:} @voodoos (Tarides), @trefis (Tarides), @Ekdohibs (OCamlPro), @gasche (INRIA), @Octachron (INRIA)

\textbf{Why:} Enhance code navigation and refactoring for developers by providing project-wide reference editor features, aligning OCaml
with the editor experience found in other languages.

\textbf{What:} Introducing \texttt{merlin single occurrences} and LSP \texttt{textDocument/references} support, extending compiler's Shapes for global
occurrences and integrating these features in Dune, Merlin, and OCaml LSP.

\textbf{Activities:}
\begin{itemize}
\item Released a preview version of project-wide references and announced it on Discuss, asking for feedback - [[ANN][PREVIEW] Play with project-wide occurrences for OCaml!](\url{https://discuss.ocaml.org/t/ann-preview-play-with-project-wide-occurrences-for-ocaml/13814})
\item Merged the compiler PR - \href{https://github.com/ocaml/ocaml/pull/12508}{ocaml/ocaml\#12506}
\item As a teaser for future work that will build on project-wide references, we started prototyping the project-wide \texttt{rename} feature - \href{https://github.com/voodoos/ocaml-lsp/tree/index-preview}{voodoos/ocaml-lsp\#index-preview}
\end{itemize}
\subsubsection*{\textbf{{[}Merlin]} Improving Merlin's Performance (\href{https://ocaml.org/docs/platform-roadmap\#w19-navigate-code}{W19})}
\label{sec:org34a301d}

\textbf{Contributed by:} @pitag (Tarides), @Engil (Tarides)

\textbf{Why:} Some Merlin queries have been shown to scale poorly in large codebases, making the editor experience subpar. Users report
that they sometimes must wait a few seconds to get the answer. This is obviously a major issue that hurts developer experience, so
we're working on improving Merlin performance when it falls short.

\textbf{What:} Developing benchmarking tools and optimising Merlin's performance through targeted improvements based on profiling and
analysis of benchmark results.

\textbf{Activities:}
\begin{itemize}
\item We merged the Fuzzy testing CI. As a reminder, this CI tests Merlin PRs for behaviour regressions. This will help us make sure that we don't inadvertently break Merlin queries by testing them on a broad range of use cases - \href{https://github.com/ocaml/merlin/pull/1716}{ocaml/merlin\#1716}
\item In \texttt{merlin-lib}, we started writing a prototype to process the buffer in parallel with the query computation. Parallelism refers to OCaml 5 parallelism (domains).
\end{itemize}
\section*{Discussions on the future of the opam repository}
\label{4}
Archive: \url{https://discuss.ocaml.org/t/discussions-on-the-future-of-the-opam-repository/13898/9}
\subsection*{Kate announced}
\label{sec:orgb892fc3}


The notes for last week's meeting are available \href{https://github.com/ocaml/opam-repository/issues/23789\#issuecomment-1961757335}{here}

\textbf{For everyone who wants to come to the next meeting, please fill \href{https://framadate.org/qD2Pb57B7h6xJ8U4}{the framadate} as soon as
you know when you are available, so that we can plan when the meeting is going to be.}
\section*{ocaml-protoc-plugin 5.0.0}
\label{5}
Archive: \url{https://discuss.ocaml.org/t/ann-ocaml-protoc-plugin-5-0-0/14205/1}
\subsection*{Anders Fugmann announced}
\label{sec:org4102034}


It's my pleasure to announce release 5.0.0 of \href{https://github.com/andersfugmann/ocaml-protoc-plugin}{ocaml-protoc-plugin}.

ocaml-protoc-plugin is a plugin for the google protobuf compiler, \texttt{protoc}, to generate type mappings and functions for serialization
and de-serialization of google protocol buffers. The plugin aims to be fully compliant with the protobuf specification and
recommendations and to generate an intuitive mapping between google protobuf message definitions and Ocaml types. Ocaml-protoc-plugin
is written in pure ocaml.

Version 5.0.0 includes option to merge messages to be fully compliant with the protobuf specification and fixes bugs related to name
mapping to avoid name collisions and fix code generation error in some corner cases.

Serialization and deserialization has also been hugely optimized for speed and is now on par with other ocaml protobuf
implementations (benched against ocaml-protoc)

ocaml-protobuf-plugin 5.0.0 is available though opam and from the \href{https://github.com/andersfugmann/ocaml-protoc-plugin}{project
page} on github.

Full changelog is available \href{https://github.com/andersfugmann/ocaml-protoc-plugin/releases}{here} 
\section*{ppx\_minidebug 1.3.0: toward a logging framework}
\label{6}
Archive: \url{https://discuss.ocaml.org/t/ann-ppx-minidebug-1-3-0-toward-a-logging-framework/14213/1}
\subsection*{Lukasz Stafiniak announced}
\label{sec:org9f8659c}


Hi! I'm happy to invite you to take a look at \href{https://github.com/lukstafi/ppx\_minidebug}{ppx\_minidebug 1.3.0}. It's now available in
the opam repository. Some new features since version 1.0:

\begin{itemize}
\item Extension point variants that support debug runtime passing, they simplify having e.g. dedicated log files for threads or domains.
\item Unregistered extension points \texttt{\%log}, \texttt{\%log\_result}, \texttt{\%log\_printbox} to explicitly log values.
\item Log levels at runtime to restrict how much is logged, and at compile time to restrict how much logging code is generated.
\begin{itemize}
\item Log levels can be both set globally and adjusted for local scopes.
\item Compile time log levels can be read off an environment variable.
\end{itemize}
\item Extension point prefix \texttt{\%diagn\_} (joining prefixes \texttt{\%debug\_} and \texttt{\%track\_}) that restricts the log level to explicit logs.
\item Does not crash for logs that escaped all log entries -- prints the entry id of the entry the orphaned log lexically belongs to.
\begin{itemize}
\item Optionally prints log entry ids for all entries.
\end{itemize}
\item Optionally snapshots unclosed log trees:
\begin{itemize}
\item outputs the current log tree if sufficient time passed since the previous tree was printed or the previous snapshot, erases the previous snapshot when snapshotting or printing the same log tree.
\end{itemize}
\item A few more changes that improve usability.
\end{itemize}

P.S. If you face problems with missing line breaks in the HTML output, re-install \href{https://github.com/c-cube/printbox}{printbox from
source} or version > 0.10.
\section*{iostream 0.2}
\label{7}
Archive: \url{https://discuss.ocaml.org/t/ann-iostream-0-2/14214/1}
\subsection*{Simon Cruanes announced}
\label{sec:org3bc182a}


Iostream 0.2 was recently released on opam. Here's the \href{https://github.com/c-cube/ocaml-iostream/releases/tag/v0.2}{release itself}.

Iostream is a library providing a small set of abstractions for I/O streams (!) over \texttt{bytes}. With release 0.2, there are now 4 main
types, all of them based on objects:
\begin{itemize}
\item \href{https://c-cube.github.io/ocaml-iostream/iostream/Iostream/In/index.html}{\texttt{Iostream.In.t}} for unbuffered inputs;
\item \href{https://c-cube.github.io/ocaml-iostream/iostream/Iostream/In\_buf/index.html}{\texttt{Iostream.In\_buf.t}} for buffered inputs;
\item \href{https://c-cube.github.io/ocaml-iostream/iostream/Iostream/Out/index.html}{\texttt{Iostream.Out.t}} for unbuffered outputs;
\item \href{https://c-cube.github.io/ocaml-iostream/iostream/Iostream/Out\_buf/index.html}{\texttt{Iostream.Out\_buf.t}} for buffered outputs.
\end{itemize}

I found out that having all these explicit types is better than picking only some of them. A \texttt{In\_buf.t} can act as a proper byte
stream, exposing its internal slice of bytes so that it's possible to implement line parsing and such. Thanks to the object types,
\texttt{In\_buf.t} is a subtype of \texttt{In.t}, and same goes for \texttt{Out\_buf.t} and \texttt{Out.t}.

There is also a \href{https://c-cube.github.io/ocaml-iostream/iostream-camlzip/Iostream\_camlzip/index.html}{\texttt{iostream-camlzip}} library
that implements stream encoding and decoding over these types.
\section*{Bogue, the OCaml GUI}
\label{8}
Archive: \url{https://discuss.ocaml.org/t/ann-bogue-the-ocaml-gui/9099/57}
\subsection*{sanette announced}
\label{sec:orgc23b8a4}


I'm happy to announce a new version of \href{https://github.com/sanette/bogue}{Bogue} (20240225). I would like to
mention here two main changes:
\begin{itemize}
\item This is the \textbf{last version that supports SDL 2.0.9} (default in Debian 10) Next version will use \texttt{tsdl 1.0.0} which requires SDL 2.0.10 or later
\item Thanks to @edwin , this version (if using SDL >= 2.0.16) is much more  \textbf{\textbf{power-friendly}} (when idle). You may now leave a Bogue app open (if idle) without worrying for your laptop battery (and the environment): energy consumption is now very close to zero. (\texttt{powertop} indicates instantaneous power of 0mW, which I think means less than 0.1mW on my laptop)
\end{itemize}
\section*{Owl project restructured}
\label{9}
Archive: \url{https://discuss.ocaml.org/t/owl-project-restructured/14226/1}
\subsection*{jrzhao42 announced}
\label{sec:org19fd6d3}


Dear OCaml Community, following our previous decision to conclude the Owl project, we have been touched by the
supportive and encouraging feedback from all of you. After a thorough discussion, Liang @ryanrhymes and I think it
might still be for the best interest of the OCaml community to continue maintaining a solid numerical computing
library. Consequently, I, Jianxin, will assume the role of project leader to ensure Owl remains maintained. Our
goal is to keep Owl stable and updated, given the very limited resource we have, as explained in our previous
declaration.  At least we aim to keep Owl compatible with the latest stable version of OCaml.

As mentioned previously, our availability to dedicate time to Owl is limited. Achieving our objectives will require
collective effort. Thus, I am looking to assemble a team of contributors eager to support both development and
maintenance tasks. For details on our plans and how you can contribute, please refer to the project's
\href{https://github.com/owlbarn/owl}{README} file.

If you're interested in joining the Owl team, taking on a specific part of the codebase, or if you have any
questions, do not hesitate to contact me here or via [email](@\url{mailto:jianxin.zhao@kit.edu}).
\section*{20+ ways to build an executable with foreign libs}
\label{10}
Archive: \url{https://discuss.ocaml.org/t/20-ways-to-build-an-executable-with-foreign-libs/14227/1}
\subsection*{Gregg Reynolds announced}
\label{sec:org1c9f146}


Foreign libs can be linked into OCaml executables in a variety of ways. The pertinent info is distributed across
several topics in the manual.  So I put together a suite of 20-some MWEs demonstrating the options:

\begin{itemize}
\item emitting native or bytecode executables
\item vm executables: freestanding or dependent on ocamlrun
\item using a static or dynamic stublib
\item linking the stublib against static or shared foreign libs
\item putting lib deps directly on the cmd line v. using -cclib or -dllib
\end{itemize}

All demos use a very simple foreign lib (in C) and stublib, and the builds are expressed in a few relatively simple
makefiles.

\url{https://github.com/obazl/demos\_obazl/tree/main/makefiles/ffi}

This could be used as the basis of a proper tutorial, but I'm not sure I'll get around to that.

(Oh crap, I forgot to update the docs. The make target names a little different. Read the makefiles. ;)
\section*{Re: opam build and opam test: the opam plugins that simplifies your dev setup}
\label{11}
Archive: \url{https://discuss.ocaml.org/t/ann-opam-build-opam-test-the-opam-plugins-that-simplifies-your-dev-setup/8867/2}
\subsection*{Kate announced}
\label{sec:orgdfe15a2}


Hi everyone,

I'm pleased to announce the release of opam-build \& opam-test 0.2.0 (soon 0.2.4 with all the latest improvements)

These two opam plugins now require your current client to be opam 2.2 (e.g. the latest 2.2.0\textasciitilde{}beta1).
If you use opam 2.2 you can install them using:
\begin{verbatim}
opam update && opam install 'opam-build>=0.2.0' 'opam-test>=0.2.0'
\end{verbatim}
After that you can use them from any switches using:
\begin{verbatim}
opam build --help
opam test --help
\end{verbatim}

The highlights of this version cycle are:
\begin{itemize}
\item Vastly improved performance and UI
\item Added a new \texttt{-{}-{}global} and \texttt{-{}-{}local} command line argument to signify whether to use a local switch or a global switch
\item Add a new config file storing the user preference and which kind of switch to use by default
\item Lots of fixes and improvements
\end{itemize}

On a personal note, my main incentive with these changes was to finally use these plugins personally and in
particular, while working on some packages for my work on the ``OCaml 5.2 release readiness'' (see a more general
description of this work
\href{https://discuss.ocaml.org/t/ocaml-software-foundation-january-2024-update/13828\#infrastructure-5}{here}), i
encountered a couple of packages where the only way to compile them was using opam as they were using custom
variables passed to their arguments. \texttt{opam install} was not what i wanted (i don't want to install it, i just need
to see if the local version compiles) but \texttt{opam build} fits the bill perfectly here (I don't know what build-system
it uses, i just want to compile it).

A demo of the new version (here 0.2.4) can be seen here:

\url{https://global.discourse-cdn.com/business7/uploads/ocaml/original/2X/a/ab3b42f5b36aac43a4107b8d288011e4688644be.gif}

\textbf{Disclaimer}: As with version 0.1.0, these plugins are still \textbf{experimental}, however they should be a lot more
polished and usable, hopefully if enough people report issues (big thanks to @gridbugs for the reports on the
previous versions) next version should be deemed stable.
\section*{Seeking feedback on repackaged libraries for Base}
\label{12}
Archive: \url{https://discuss.ocaml.org/t/seeking-feedback-on-repackaged-libraries-for-base/14241/1}
\subsection*{Mathieu Barbin announced}
\label{sec:org6ef0ce0}


Dear OCaml community,

Lately I've been working on reducing dependency footprints in some of my projects, which led me to repackage some
of Jane Street's libraries that I frequently use. The goal was to make these libraries depend solely on \texttt{Base},
thereby making them more accessible and lightweight.

Here are the libraries I've repackaged so far:

\begin{itemize}
\item \href{https://github.com/mbarbin/appendable-list}{Appendable List}
\item \href{https://github.com/mbarbin/doubly-linked}{Doubly Linked List}
\item \href{https://github.com/mbarbin/nonempty-list}{Nonempty List}
\item \href{https://github.com/mbarbin/reversed-list}{Reversed List}
\item \href{https://github.com/mbarbin/union-find}{Union Find}
\end{itemize}

These libraries required minimal modifications to remove the original dependencies from \texttt{Core}, \texttt{Core\_kernel}, or
\texttt{Core\_extended}, depending on the case.

For instance, the \texttt{Nonempty List} library is a repackaged version of \texttt{Core\_kernel}'s \texttt{Nonempty\_list}. The original
code can be found \href{https://github.com/janestreet/core\_kernel/tree/master/nonempty\_list}{here}. I've modified the
code slightly to remove dependencies on \texttt{Core} and \texttt{Core\_kernel}, making it solely dependent on \texttt{Base}. This allows
the library to be used in more contexts without the need to add a dependency on \texttt{Core} and \texttt{Core\_kernel}.

I'm reaching out to gauge interest in these repackaged libraries. While the use case might be niche, I believe
there could be \texttt{Base} users who are motivated enough to limit their dependencies on \texttt{core}, etc.

If there's interest in the community for these libraries, I'm considering reaching out to Jane Street to discuss
options. If Jane Street isn't interested but the community is, I'm open to moving these libraries into a community
space. I'm also willing to volunteer as a maintainer in this case. Of course, I want to respect the community's
established practices and wouldn't want to step on anyone's toes.

If there's no interest, I'm happy to continue as is. I'm simply offering this in case it might be beneficial to
others.

I appreciate your time and look forward to your feedback!
\section*{Outreachy internship demo session}
\label{13}
Archive: \url{https://discuss.ocaml.org/t/outreachy-internship-demo-session/14247/1}
\subsection*{Fay Carsons announced}
\label{sec:orgf570690}


Hi all! Myself and the other Outreachy interns from this batch are going to be publicly demoing our projects
{[}date=2024-03-06 time=12:00:00 timezone=``America/New\_York''] !

Im going to be talking about \emph{Joy}, the creative coding library I developed, going over some features and hopefully
doing some live-coded generative art! The other interns will be presenting their work with \emph{Bogue} and refactoring
the UI of Ocaml.org

Stop by if you'd like!
\href{https://meet.google.com/rym-eqax-uwb?hs=122\&authuser=0}{link}
\section*{dream-html 3.0.0}
\label{14}
Archive: \url{https://discuss.ocaml.org/t/ann-dream-html-3-0-0/14013/5}
\subsection*{Yawar Amin announced}
\label{sec:orgea17c5a}


{[}ANN] dream-html 3.1.0

Thanks to the efforts of Kento Okura, I am happy to announce the release of
\href{https://ocaml.org/p/dream-html/3.1.0}{3.1.0}, which brings complete support for all standard
\href{https://developer.mozilla.org/en-US/docs/Web/MathML}{MathML} attributes and elements.

Eg:

\begin{verbatim}
open Dream_html
open MathML

let op sym = mo [] [txt "%s" sym]

let pow i n = msup [] [
  mi [] [txt "%s" i];
  mn [] [txt "%s" n];
]

(* a^2+b^2=c^2 *)
let pythagoras_theorem = mtable [] [
  mtr [] [
    mtd [] [pow "a" "2"; op "+"; pow "b" "2"];
    mtd [] [op "="];
    mtd [] [pow "c" "2"];
  ];
]
\end{verbatim}

As you can see, we can write helpers that greatly reduce duplication.

This addition completes dream-html's support for all standard XML-based markups that are rendered by browsers.

This release also deprecates a couple of non-standard HTML attributes that I had mistakenly added before.
\section*{Add your OCaml Events to the Community Page on OCaml.org}
\label{15}
Archive: \url{https://discuss.ocaml.org/t/add-your-ocaml-events-to-the-community-page-on-ocaml-org/14251/1}
\subsection*{Sabine Schmaltz announced}
\label{sec:org4085b9d}


Hey folks!

This is a call for anyone who is running or knows about upcoming OCaml-related events to add those events to the
Events directory on OCaml.org.

Here's how to do it:

Open a pull request similar to this one:

\url{https://github.com/ocaml/ocaml.org/pull/2134}

If it's a recurring event, create a listing in \texttt{recurring.yml}. If it's a one-time event, you should omit this.

If there's start or end times, they need to be given in UTC - so that we can, in a later improvement to the Events
directory, convert them to the viewer's local timezone more easily. Here's an example of an event with a start
time:

\begin{verbatim}
---
title: "OCaml Users in Paris (OUPS)"
textual_location: Paris, France
location:
  lat: 48.8566
  long: 2.3522
url: https://www.meetup.com/ocaml-paris/events/299014082/
recurring_event_slug: ocaml-users-paris-oups
starts:
  yyyy_mm_dd: "7024-02-29"
  utc_hh_mm: "18:00"
---
\end{verbatim}
\section*{Other OCaml News}
\label{16}
\subsection*{From the ocaml.org blog}
\label{sec:org64b0ede}


Here are links from many OCaml blogs aggregated at \href{https://ocaml.org/blog/}{the ocaml.org blog}.

\begin{itemize}
\item \href{https://frama-c.com/fc-versions/nickel.html}{Release of Frama-C 28.1 (Nickel)}
\item \href{https://tarides.com/blog/2024-02-28-two-major-improvements-in-odoc-introducing-search-engine-integration}{Two Major Improvements in odoc: Introducing Search Engine Integration}
\end{itemize}
\section*{Old CWN}
\label{sec:org42fe5a7}
If you happen to miss a CWN, you can \href{mailto:alan.schmitt@polytechnique.org}{send me a message} and I'll mail it to you, or go take a look at \href{https://alan.petitepomme.net/cwn/}{the archive} or the \href{https://alan.petitepomme.net/cwn/cwn.rss}{RSS feed of the archives}.

If you also wish to receive it every week by mail, you may subscribe to the \href{https://sympa.inria.fr/sympa/info/caml-list}{caml-list}.

\begin{authorname}
\href{https://alan.petitepomme.net/}{Alan Schmitt}
\end{authorname}
\end{document}
